%!TEX root=./main.tex
\usepackage{hyperref}
\usepackage{xspace}
\usepackage[dvipsnames]{xcolor}
% \usepackage{amsmath}
% \usepackage{amssymb}
% \usepackage{amsmath}
% \usepackage{subcaption}
% \usepackage{import}
% \usepackage{commath}
% \usepackage[capitalise]{cleveref}
% \usepackage{algorithm}
% \usepackage{algpseudocode}
% \usepackage{varwidth}
\usepackage{amsopn}
\usepackage[binary-units]{siunitx}
\usepackage[normalem]{ulem}
\usepackage{multirow}

\usepackage{tikz}
\usepackage{pgfplots} 
\usepackage{pgfgantt}
\usepackage{pdflscape}
\usepackage{relsize}
\pgfplotsset{compat=newest} 
\pgfplotsset{plot coordinates/math parser=false}

\makeatletter
\DeclareRobustCommand\onedot{\futurelet\@let@token\@onedot}
\def\@onedot{\ifx\@let@token.\else.\null\fi\xspace}
\DeclareRobustCommand\nodot{\futurelet\@let@token\@nodot}
\def\@nodot{\ifx\@let@token.\else~\null\fi\xspace}

\newfont{\eaddfnt}{phvr8t at 12pt}
\def\email#1{{{\eaddfnt{\par #1}}}}

\def\eg{\emph{e.g}\onedot} \def\Eg{\emph{E.g}\onedot}
\def\ie{\emph{i.e}\onedot} \def\Ie{\emph{I.e}\onedot}
\def\cf{\emph{c.f}\onedot} \def\Cf{\emph{C.f}\onedot}
\def\etc{\emph{etc}\onedot} \def\vs{\emph{vs}\onedot}
\def\na{\emph{n.a}\onedot} \def\NA{\emph{N.A}\onedot}
\def\wrt{w.r.t\onedot} 
\def\wlg{w.l.o.g\onedot} 
\def\dof{d.o.f\onedot}
\def\aka{a.k.a\onedot}
\def\etal{\emph{et al}\onedot}
\makeatother


\newcommand{\CC}{C\nolinebreak\hspace{-.05em}\raisebox{.2ex}{\relsize{-1}{\textbf{+}}}\nolinebreak\hspace{-.10em}\raisebox{.2ex}{\relsize{-1}{\textbf{+}}}\xspace}

% \newcommand{\fig}[1]{\mbox{Figure \ref{#1}}\xspace}
\newcommand{\fig}[1]{\mbox{\figurename\xspace\ref{#1}}}
% \newcommand{\subfig}[2]{\mbox{\cref{#1}#2}\xspace}

%reference to Table with mbox
\newcommand{\tab}[1]{\mbox{Table \ref{#1}}\xspace}

%reference to Section with mbox 
\newcommand{\sect}[1]{\mbox{\S\ref{#1}}\xspace}
% \newcommand{\sect}[1]{\mbox{\cref{#1}}\xspace}

\newcommand{\eq}[1]{\mbox{\eqref{#1}}}

% \usepackage{amsmath}

% \DeclareMathOperator*{\argmax}{arg\,max}

%use this one for the final version to show all the changes in red
% \newcommand{\textcomment}[3]{\textcolor{red}{#3}}
\newcommand{\textcomment}[3]{\textcolor{#2}{#1: #3}}
\newcommand{\SG}[1]{\textcolor{RawSienna}{#1}}
\newcommand{\SGC}[1]{\textcomment{SG}{olive}{#1}}
\newcommand{\AB}[1]{\textcomment{AB}{purple}{#1}}
\newcommand{\LC}[1]{\textcomment{LC}{ForestGreen}{#1}}
\newcommand{\DP}[1]{\textcomment{DP}{WildStrawberry}{#1}}
\newcommand{\TC}[1]{\textcomment{TC}{Periwinkle}{#1}}
\newcommand{\todo}[1]{\textcomment{TODO}{red}{#1}}

\DeclareSIUnit\inch{in}

\newcommand{\bmat}[1]{\ensuremath{\begin{bmatrix} #1 \end{bmatrix}}}
% Vector: print the argument as a vector
\newcommand{\vet}[1]{\ensuremath{\mathbf{#1}}}
% normal vector
\newcommand{\nvet}[1]{\ensuremath{\hat{\vet{#1}}}}
% Direction vector
\newcommand{\dvet}[1]{\ensuremath{\overrightarrow{\vet{#1}}}}

% Matrix: print the argument as a matrix
\newcommand{\mat}[1]{\ensuremath{\mathtt{\uppercase{#1}}}}

% Inverse: print a -1 on the top right of the argument 
\newcommand{\inv}[1]{\ensuremath{{#1}^{\text{-}1}}}

% Inverse: print a -1 on the top right of the argument 
\newcommand{\minv}[1]{\ensuremath{\mat{{#1}}^{\text{-}1}}}

% Transpose: print a T on the top right of the argument 
\newcommand{\tra}[1]{\ensuremath{{#1}^{\mathsf{T}}}}

% Transpose Matrix: print a T on the top right of the argument intended to be a matrix 
\newcommand{\mtra}[1]{\ensuremath{\tra{\mat{#1}}}}

% Transpose Vector: print a T on the top right of the argument intended to be a vector
\newcommand{\vtra}[1]{\ensuremath{\tra{\vet{#1}}}}

% minus transpose:  print a -T on the top right of the argument
\newcommand{\ment}[1]{\ensuremath{{#1}^{\text{--}\mathsf{T}}}}

% Cross Matrix:  print the argument in the cross matrix notation
\newcommand{\crmat}[1]{\ensuremath{\left[{#1}\right]_{\times}}}


\newcommand{\termName}[2]{\ensuremath{#1_{\text{#2}}}\xspace}
\newcommand{\Einternal}{\termName{E}{int}}
\newcommand{\Econtour}{\termName{E}{con}}
\newcommand{\Epoint}{\termName{E}{point}}
\newcommand{\lambdaContour}{\termName{\lambda}{cont}}
\newcommand{\lambdaInternal}{\termName{\lambda}{int}}
\newcommand{\dplane}{\termName{d}{plane}}
\newcommand{\thetaTPS}{\termName{\theta}{TPS}}
\newcommand{\wTPS}{\termName{w}{TPS}}
\newcommand{\xref}{\termName{\vet{x}}{ref}}

