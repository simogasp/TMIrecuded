%!TEX root=./main.tex

\section{Related Work}
\label{sec:sota}
\subsection{Scope}
The types of AR-guided laparosurgery with the most clinical impact involve the fusion of \emph{pre-operative} medical image data from MR or CT, to which we therefore limit the scope of this section.
A broader perspective, including fusion with intra-operative images such as Cone Beam CT (CBCT), is given in~\cite{Bernhardt2017}.
An important categorization of approaches is whether they work with monocular~\cite{affineTracking} or stereo laparoscopes~\cite{21142942,conf/miccai/Amir-KhaliliNPHA13,Cohen2010Prostate,hamarneh2014igrs,haouchine13,Su2009, MaierHein2013}.
As ours is in the former category, we mostly focus on that category.
Any approach that works with monocular scopes can be applied to stereo scopes. The converse is not true, because existing methods using stereo laparoscopes require depthmaps obtained by stereo triangulation~\cite{DBLP:conf/miccai/StoyanovSPY10}, which are not available with monocular images.
\SG{Recently, Convolutional Neural Networks (CNNs) have been trained to recover depth from endoscopic images (colonoscopy~\cite{Liu2018} and bronchoscopy~\cite{Visentini2017}), however they are yet unsuccessful with laparoscopy because of the large variability in image content.}
The availability of depthmaps fundamentally changes a registration problem, because they provide 3D-to-3D registration constraints.
This contrasts with monocular registration, where 3D-to-3D constraints are unavailable.
Like us, previous approaches solve monocular registration in two stages: an initial registration stage and a tracking stage.

\subsection{Initial Registration}
%Previous work can be broadly divided into two categories depending on whether the organ model to be registered is acquired intra-operatively in a hybrid operating room, or pre-operatively in a standard operating room. 
%The first category involves determining a rigid transform between the sensors' 3D coordinate frames. 
%In laparosurgery some solutions have been proposed by either externally tracking the laparoscope with optical or magnetic markers~\cite{Su2009,Liu2016Laparoscopic} or placing on tissue artificial markers that can be detected in both modalities~\cite{Simpfendrfer2011}. 
%If additional interventional modalities are not available or not sufficiently informative then AR can be performed with pre-operative modalities. 
%In this case the registration problem can be much more challenging due to soft tissue deformation between acquisition times. 
%Also, once significant changes are made during surgery the pre-operative data becomes ``outdated'', and less useful. 

Despite considerable research, there exists no automatic and robust solution to the initial registration with a soft-body organ. State-of-the-art approaches tend to be organ-specific and have mainly focused on registering the liver~\cite{haouchine13,haouchine:hal-01186011,plantefeve:hal-01205194,Robu2018,Reichard2017}, the prostate~\cite{Cohen2010Prostate} and the kidney~\cite{Su2009,nosrati2014simultaneous,affineTracking}.
So far the only existing approaches for the initial registration with monocular images require a manual registration~\cite{affineTracking} and an interactive Graphical User Interface (GUI), which is not practical in real OR conditions.
Of the stereo methods, some perform registration with a manual GUI~\cite{Cohen2010Prostate,haouchine13} and others perform it semi-automatically with manually located landmarks~\cite{21142942,conf/miccai/Amir-KhaliliNPHA13,hamarneh2014igrs,Su2009} \SG{or using 3D surface features~\cite{Reichard2017}}.
In some works the registration is refined by Iterative Closest Point (ICP)~\cite{hamarneh2014igrs,Su2009}. % and in other works tracking is performed with texture features~\cite{haouchine13}.
The initial registration requires non-visual constraints to prevent unlikely or physically implausible deformations. Various models have been used, including rigidity~\cite{Su2009}, deformation smoothness with 3D splines~\cite{conf/miccai/Amir-KhaliliNPHA13} and bio-mechanics~\cite{hamarneh2014igrs,haouchine13,Reichard2017}.
There is no general consensus on the best model to use, as it depends heavily on available boundary conditions, available knowledge of mechanical tissue properties, and computational resources.
\SG{Recent works have built on deep learning~\cite{Brunet2019,Pfeiffer2019,Mahmood2018}, showing promising results on synthetic data only.}

A general limitation of the previous works is that they only use one monocular or one stereo image pair to constrain the initial registration.
This is limiting because registration accuracy depends strongly on how much organ surface is visible. 
This can be very small, particularly for larger organs such as the liver and uterus, leading to poor registration.

%For the stereo methods a partial 3D reconstruction is computed (known as a \emph{2.5D reconstruction} in computer vision), and all regions of the organ not visible in both images have no 3D information. 
%Thus the registration is essentially guessed at these regions from the deformation model's prior. 
%Secondly, they do not use the organ's \emph{occluding contours} as registration constraints. Typically a stereo reconstruction never computes 3D information well at the occluding contours. 
%The occluding contours provide powerful boundary conditions and should be exploited whenever possible.

% \SG{our~\cite{Collins2017System}}

\subsection{Tracking}
\label{sec:sotaTracking}
Almost all monocular approaches rely on the detection and tracking of features, either artificial fiducial markers~\cite{Cohen2010Prostate} inserted on the organ, or natural keypoints. The former are invasive and generally not practical. The latter are sensitive to illumination changes, large camera motion and occlusion. These factors critically affect the performance of tracking as they restrain the capability of maintaining the registration, and hence AR visualization, for long periods of time, especially if the organ is deformed or occluded by, \eg, the surgery tools. 

To date, only one previous work has been capable of robust long duration tracking of the kidney (several minutes) without artificial fiducials~\cite{affineTracking}. 
 %Similarly to~\cite{affineTracking}, we rely on an initial registration to align the model and project features onto the model, required for the tracking phase. 
This work has however two main limitations. 
Firstly, only one reference image is used, which means features only exist on the surface region visible in the reference image. 
Tracking therefore breaks down if the organ is seen from strong viewpoint changes. This is a common situation for the uterus, because unlike the kidney it is highly mobile, and is often moved by the surgeon's assistant with a cannula. 
Secondly, the initial registration is performed manually, which is not practical in real OR conditions. 
In our approach, we overcome both of these limitations. 

%to several reference laparoscopic images (only one in~\cite{affineTracking}). 
%Then, instead of using Shi-Tomasi corners~\cite{Shi1994Good} as~\cite{affineTracking}, we detect SIFT features~\cite{Lowe:2004:DIF:993451.996342} in the reference images and we map them onto the model's surface. 
%Once done, the model is automatically registered to new laparoscopic images using feature-based tracking. 
%An important difference is that the initial registration was done manually with a rigid model in~\cite{affineTracking}, whereas we propose a semi-automatically registration with a deformable model. 
%Therefore we can handle deformation due to insufflation and other factors, which is required for accurate registration of soft organs. 


% The only systems based on natural features that are capable of robust registration over long durations are~\cite{Collins2044} for the uterus and~\cite{affineTracking} for the kidney.
% They both rely on an initial registration, which aligns the model to one~\cite{affineTracking} or several~\cite{Collins2044} reference laparoscopic images. 
% Then 2D texture features, Shi-Tomasi corners~\cite{Shi1994Good} and SURF~\cite{SURF} respectively, are detected in the reference images and mapped onto the model's surface. 
% Once done, the model is automatically registered to new laparoscopic images using feature-based tracking. 
% An important difference between~\cite{affineTracking} and~\cite{Collins2044} is that in~\cite{affineTracking} the initial registration was done manually with a rigid model, whereas in~\cite{Collins2044} it was done semi-automatically with a deformable model. 
% Therefore~\cite{Collins2044} could handle deformation due to insufflation and other factors, which is required for accurate registration of soft organs. 



% The uterus is a flexible organ that can exhibit strong deformation when manipulated with laparoscopic tools~\cite{DBLP:conf/ipcai/MaltiBC12}.
% However, when observing the uterus during intervention prior to resection it remains quite rigid does not deform significantly due to respiration. %We target the problem of registering the uterus before resection begins. The objective is to use our solution as the foundation for applying AR to assist intra-operative resection planning. 
% Optical registration in laparoscopy has been studied previously for other organs using the assumption of rigid, or approximately rigid motion. 
% This has been developed with monocular~\cite{Davison:2007:MRS:1263144.1263479,Garcia:etal:AMIARCS09,DBLP:conf/icra/GrasaCM11} and stereo laparoscopes~\cite{DBLP:conf/miccai/MountneySDY06,DBLP:conf/miccai/TotzMSY11}. 

Markerless tracking is also addressed by visual Simultaneous Localisation and Mapping (SLAM)~\cite{Thrun2002Robotic,Mahmoud2017}. 
%Visual SLAM relies only on raw optical data, and does not need other hardware such as magnetic~\cite{Nakamoto2008, Liu2016Laparoscopic, Xiao2018} or optical~\cite{Simpfendrfer2011} tracking devices. 
In SLAM,  a 3D representation or \textit{map} of the environment is incrementally built and updated and, at the same time, the camera is localized \wrt the map.
% SLAM involves building a 3D representation of the environment, known as the \textit{map}, and computing the rigid transform which positions the map in the camera's coordinate frame. 
%The core challenge in SLAM is how to achieve \textit{data association}. 
%SLAM requires data association in two respects. 
%The first is for \emph{map building}. %This is to determine which points from different video frames correspond to the same surface point. Solving this problem allows us to construct a map of the points in 3D space. 
%The second is for \emph{localisation}, which is to determine where the map's points are located in a new input image. 
%SLAM offers a fast solution to these problems and has found considerable success in man-made environments.
However, SLAM in laparoscopy is not reliable enough for routine clinical use. %This is due to the repeated nature of tissue texture, rapid camera motion and photo-constancy violations caused by blood, strong viewpoint, or strong illumination change for example. 
This is because monocular SLAM systems assume a rigid scene and hence are incapable of tracking a mobile organ such as the uterus, as we show in \sect{sec:orbslam}.
A deformable SLAM method has been recently introduced~\cite{Lamarca2019}. 
The principle is promising but the method requires that the scene is made of a single deforming object. 
%faces considerable challenges when the scene is not globally rigid.
%When the scene is made up of one or more independently moving structures, such as the uterus, SLAM can make errors by merging features from different structures into one map. 
%For laparoscopic procedures involving the uterus, a typical scene will comprise the uterus, ovaries, peritoneum, small intestine, and bladder. 
%In most procedures, a cannula is inserted into the uterus through the vagina and is operated externally
%by an assistant. 
%The assistant's hand movement causes the uterus to move independently of the surrounding structures. 

%One problem is to ensure the map comprises features from the uterus and not background structures. 
%Thus, this requires an additional segmentation step that computes binary masks labelling pixels as either being on the uterus body or not. 
%A possibility would be to connect CNN based uterus segmentation with SLAM but we have not found such an approach in the literature.
% However, achieving this automatically is difficult and has not been studied in the literature. \SG{maybe some recent DL/CNN?}
%A naive way to proceed would be to mask the uterus manually in one or more frames and enforce that SLAM uses features found only within the masks. 
%However, there is no guarantee that SLAM will not eventually use features from surrounding organs, thus leading to mapping and localization errors. 
%By contrast, it is infeasible to mask frames manually for every frame.

%In this work, instead, we solve the registration problem using a \emph{tracking-by-detection} paradigm, in which each frame is processed independently from one to another (\ie no feature tracking over the frames).
%After the initial registration of the pre-operative 3D model and the MVS reconstruction from laparoscopic images, we align the MVS model and the current frame my matching the features extracted in the MVS step and the ones of the current frame.
%This is proved to be robust to occlusions that prevent the correct tracking of the features. 
%This also avoids the problem of background structures as we are matching the features only \wrt the features of the MVS model, thus limiting the manual segmentation only to the initial, off-line MVS reconstruction step. 

\subsection{Contributions}
\label{sec:contributions}
This work describes the first complete pipeline to provide AR-guided uterine laparoscopic surgery without artificial markers or tracking equipment. It is therefore strongly compatible with existing workflows and hospital equipment. To achieve this we present technical innovations that overcome the limits of previous works.% and allow for robust long-duration tracking of a mobile organ whose motion can be independent of surrounding structures.
%The main clinical use cases are for AR-assisted tumor mass localization and resection planning of myomas (uterine fibroids) and adenomyomas. We target the most common clinical setting, using pre-operative 3D images and a standard handheld monocular laparoscope. There is no need for additional equipment such as optical or electromagnetic tracking systems. It is therefore strongly compatible with existing workflows and hospital equipment. 

This paper is a distillation and extension of contributions from three workshop papers~\cite{Collins2044,Collins2013,Collins2017System}. Four significant landmark results have been achieved for the first time in laparoscopic AR in this work, which we now summarize. Firstly, we show for the first time that dense \textit{invivo} 3D reconstruction is achievable with a monocular laparoscope with Structure-from-Motion (SfM) and multi-view stereo (MVS). Following this, SfM and MVS have been applied by other groups in related problems. Reconstruction is provided up to an unknown scale factor. %, which is the result of an ambiguity between a smaller scene viewed up close with smaller camera motion, compared to a larger scene viewed further away with larger camera motion. 
This ambiguity is always present in motion-based methods, including SfM and SLAM. %This is a problem because without knowing the reconstruction's absolute scale, we cannot correctly register the model. 
Secondly, we recover the reconstruction's absolute scale, by solving it simultaneously with the initial registration via numerical optimization. %Specifically, scale is constrained by physical properties of the organ's deformable model, and the problem is solved by jointly registering the model while simultaneously optimizing the reconstruction scale factor. 
Thirdly, we show how the organ's silhouette (contours in the laparoscopic images corresponding to the organ's boundary) can be used to significantly improve the initial registration. This  fruitfully complements information from the scene reconstruction (\fig{fig:initialRegOverview}),  and is particularly important for organs that are very smooth and lack strong geometric detail, such as the uterus. Without contours, the model can incorrectly slide over the reconstruction and drift from the correct solution. Fourthly, we track a mobile organ (one that, like the uterus, can move independently of surrounding structures) using robust keypoint matching within tracking-by-detection. This allows the organ to be tracked over long durations (several dozens of minutes) in the presence of difficulties including partial views, occlusions and when the organ moves out of the laparoscope's field-of-view.

\SG{This paper presents several improvements of our complete AR pipeline towards clinical use and to reduce manual effort during the initial registration.
Specifically, four main contributions considerably improved our previous work.
We (\emph{\textrm{i}}) ease  contour marking by using a touchscreen requiring only non-precise finger strokes, thus making the approach much faster and more practical in the OR.
We (\emph{\textrm{ii}}) generalize our approach to all deformable models of interest, including mechanical models such as Finite Element Models (FEMs).
We (\emph{\textrm{iii}}) propose a semi-automatic mechanism to collect the frames for 3D reconstruction of the organ by SfM.
The tool helps and guides the surgeon in the exploratory phase to acquire sharp images that are also sufficiently spread in space.
We show that this has a dramatic impact on the reconstruction quality and tracking robustness.
We (\emph{\textrm{iv}}) considerably improve the tracking algorithm to work with a robust real-time SIFT detector, which sensibly increases tracking robustness and smoothness.
We present a manual mechanism to update the model's keypoints over time to overcome small appearance and geometric changes in the organ during tracking.
This allows us to handle gradual texture changes of the organ during surgery and significantly improves tracking quality.
We compare tracking performance against the state-of-the-art in real-time monocular laparoscopic tracking (ORB-SLAM~\cite{orbslam_laparo}).}

The result is a system that we have tested $12$ times live in the OR, which we recall has not been previously achieved for any organ, including the uterus, using monocular laparoscopes and a markerless approach.




%and provide four main technical extensions, which we now summarize. %Concerning the initial registration solution, the main extensions on~\cite{Collins2044} are three-fold. 
%Firstly, the deformable model used in the previous approach was a 3D affine model. We extend the approach to work with all deformable models of interest, including mechanical models. Secondly, the organ model geometry in~\cite{Collins2044} was required to be topologically equivalent to a disc. This required virtually cutting the uterus model at the cervix. We eliminate this requirement, so the approach is compatible with arbitrary organ geometries generated by automatic 3D segmentation or interactive segmentation using, \eg, ITKSnap.
%Thirdly, for the first time, silhouette contours were proposed as a way to constrain organ registration with laparoscopic images. As we show, these are important to fixate the model during the initial registration. They are also important to help resolve the reconstruction's absolute scale, which is unknown, and we achieve this simultaneously with determining the initial registration. To use contours, we require their detection in an image. We have considerably sped up the process with an interactive user interface, requiring only non-precise finger strokes with a touchscreen interface, thus making the approach more practical in the OR setting. %Fourthly, our approach requires dense in-vivo 3D reconstruction, previously made with proprietary Structure-from-Motion software Photoscan~\cite{AgiSoft2014}. Here we show for the first time that it is possible to densely reconstruct in-vivo environments using a free open-source library ~\cite{AliceVision}, which overcomes licensing barriers for broader use in the research community. Concerning the tracking algorithm, 
%Fourthly, we propose to use a real-time SIFT implementation which substantially increases tracking robustness compared to SURF, as we show in \sect{sec:siftvssurf}. This includes a mechanism to update features over time, allowing to handle gradual texture changes of the organ during surgery. 

%We emphasize that the registration pipeline has been developed for the uterus, but it is sufficiently general to adapt to other organs, such as the kidney, most deformation models, other pre-operative modalities, and use with stereo scopes.



