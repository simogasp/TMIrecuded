%!TEX root=./main.tex

\subsection{Real-time Tracking-by-Detection}
\label{sec:updateRegistration}
\subsubsection{Overview}
Once the initial registration is computed, the real-time tracking stage starts from the live feed of the laparoscope. 
Our approach updates the initial registration at each frame and is robust to common challenges including occlusions from \eg, surgical tools, partial views and viewpoint changes.
% Having solved the initial registration we initiate the tracking stage, which updates the initial registration in real-time using live laparoscopic images. %First we extract features in the keyframe images, then matching them to each new image in a RANSAC framework. 
% This approach is robust to common challenges including occlusions from \eg, surgical tools, partial views and viewpoint changes.

%Unlike SLAM-based tracking methods it does not use frame-to-frame tracking. Instead it performs \emph{tracking-by-detection}. This allows it to keep the registration over long durations and can trivially recover when the organ is not visible for certain periods, such as when the surgeon removes and then reinserts the laparoscope or cleans the lens.
%In cases when the organ is assumed to be fixed relative to background structures, we can track using features from both the organ and background structures.
%This improves stability if the organ's texture is weak, and we do this in the ex-vivo user study. 

\subsubsection{Feature mapping}
\label{sec:preparing}
For each keyframe, we render the 3D model and we store the depth map of all the pixels lying within the model's silhouette. %We use $(x,y,z)=rim(i,j)$. to be the 3D position in camera coordinates at pixel location $(i,j)$. If $x=y=z=\mathrm{nan}$ then pixel $(i,j)$ dose not lie within the silhouette of the model, otherwise it dose. We denote $\mat{M}_{i}\in SE_{4}$ to be the rigid transform mapping the model to the $i^{th}$ reference image's frame. We can compute the 3D position of an image point lying with the model's silhouette by $\mat{M}_{i}^{-1}[x,y,z,1]^{\top}$. 
%The RIM is convenient because for each texturemap image we can extract 2D image features \textit{anywhere} within the model's silhouette, 
From this, the 3D position \vet{u} of any 2D image feature located within the model's silhouette can be determined.
%Note that without computing a dense 3D model this is not possible in general.
For each keyframe, we then extract a set of image features and to each one of them we associate its corresponding depth. 
We concatenate the features from all images into a single list $\mathcal{F}=\{(\vet{u}_{m},i_{m},\vet{d}_{m})\}$, where $\vet{u}_{m} \in \mathbb{R}^3$
% denotes the $m^{th}$ feature's 3D position in the model coordinate frame, $I_{m}$ denotes the index of the keyframe from which it was detected and $\vet{d}_{m}$ denotes its descriptor.
is the 3D point associated to feature detected in the keyframe of index $i_{m}$, with $\vet{d}_{m}$ its associated descriptor.

The approach can be used with any feature detector, such as SIFT~\cite{Lowe:2004:DIF:993451.996342}, ORB~\cite{orbslam_laparo} or SURF~\cite{SURF}. We find that SIFT features are superior in general, and they have been shown many times to obtain a better accuracy and robustness in image matching~\cite{Tuytelaars2007}. We used the recent open-source GPU implementation PopSift~\cite{Griwodz2018Popsift} to achieve real-time computation: for a typical HD $1920\times1080$ image of the uterus, up to $3000$ features can be found, taking around $\SI{11}{\milli\second}$ with a standard GPU.
%We use the green channel to compute features, rather than the standard approach of using average intensity.
%The reason is that the green light penetrates human tissue superficially and do not exhibit as much sub-surface scattering as with red light.
%The difference is very prominent with the uterus~\cite{DBLP:conf/ipcai/CollinsB12}.
To significantly reduce the problem of wrongly tracking specularities, we detect saturated pixels with an intensity threshold of $250$ and any feature within $\SI{5}{pixels}$ to a saturated pixel is discarded.  

\subsubsection{Pose Estimation Overview}
\label{sec:registration}
For each new image we extract a set of features  $\mathcal{G}=\{(\vet{y}_{i},\tilde{\vet{d}}_{i})\}$ as a list of pairs where $\vet{y}_{i}$ is the 2D image point of the features and $\tilde{\vet{d}}_{i}$ its descriptor. Our registration scheme breaks down into three main steps. First, a set of candidate matches between $\mathcal{F}$ and $\mathcal{G}$ is computed and then a pose hypothesis that best explains these matches is searched. Finally the best hypothesis for the pose is refined with the Levenberg-Marquardt (LM) algorithm by minimizing the reprojection errors.

\subsubsection{Computing Candidate Matches}
Good candidate matches of the sets $\mathcal{F}$ and $\mathcal{G}$ are those pairs with \textit{(i)} strong descriptor agreement (\ie low descriptor distance) and \textit{(ii)} a low likelihood of being false.
We achieve the latter condition by applying Lowe's Ratio Test (LRT)~\cite{Lowe:2004:DIF:993451.996342}.
%For each member of $\mathcal{F}$ we compute the member in $\mathcal{G}$ with the nearest descriptor.
%If this descriptor distance is greater than $\tau$ times the distance to the second nearest descriptor in $\mathcal{G}$, it is deemed a candidate match.
%The LRT is a very standard condition in feature-based pose estimation.
A novelty of using \emph{multiple} keyframes is that we can also exploit match \emph{coherence}. Specifically, consider a feature in the image that matches a feature in the $i^{th}$ keyframe. It is likely to be correct if there exists other features that also match with features in the $i^{th}$ keyframe.
%Enforcing coherence can reduce false matches because it prevents matches occurring from wildly different keyframe.
We adopt a ``winner-takes-all'' strategy to enforce coherence and reduce false matches.
Let $i^*$ be the index of the keyframe with the largest amount of candidate matches.
Since SIFT is invariant to scale changes and image rotation, the keyframe $i^*$ can be considered as the visually ``closest'' to the input image.
% We enforce coherence with a winner-takes-all strategy. We first find the index $I^*$ of the keyframe with the largest amount of candidate matches. This indicates the keyframe image which is ``closest'' to the input image \wrt the viewpoint. Because SIFT is invariant to scale changes and image rotation, %close means a keyframe image which views the uterus from a similar viewpoint, up to a change in depth and a rotation of the laparoscope about its optical axis.
we then recompute the candidate matches, but using \emph{only} features from the keyframe $i^*$.
Computational efficiency can be easily achieved as $\mathcal{F}$ is completely pre-computed and the distances to evaluate the descriptor likelihood are quite fast to compute on the GPU (\eg, $\sim\SI{3}{ms}$ to match $2000$ features from an HD image against $5600$ features from $16$ keyframes).
% Performing these processes is very quick because $\mathcal{F}$ is completely pre-computed, and evaluating descriptor distances can be distributed trivially on the GPU (\eg, $\sim\SI{3}{ms}$ to match $2000$ features from an HD image against $5600$ features from $16$ keyframes).

\subsubsection{Computing 3D Pose}
% Given the set of candidate matches, we perform RANSAC~\cite{Fischler:1981:RSC:358669.358692} to find the most compatible rigid 3D pose, with OpenCV's implementation with default parameters. 
From the set of candidate matches, we find the best pose hypothesis that explains these  matches using   RANSAC~\cite{Fischler:1981:RSC:358669.358692} from OpenCV's default implementation.
In some images tracking may be impossible or unreliable, if the organ is not visible or very partially. A good indicator for deciding if pose can be estimated reliably is the number $n_c$ of inlier matches found by RANSAC. If this is below a threshold (default of $8$), we reject the pose and consider the organ to be untrackable in that image. Finally, to reduce jitter artefacts, we feed the pose estimates into an Extended Kalman Filter (EKF). 

\subsubsection{Adding New Keyframes}
The operator can manually add new keyframes to $\mathcal{F}$ on the fly during the live tracking.
This mitigates any remaining jitter and stabilizes tracking if the organ is viewed from a viewpoint that is very dissimilar to the keyframes.
\SG{New keyframes also stabilize the jitter when changes occur between the reference and current peroperative states in organ appearance (bleeding spots, change of texture because of the clamping) and shape (due to mobilization).}
%In the supplementary material we present a video showing the benefits and the impact of adding a new keyframe to the tracking.
We perform this by adding the features of the current image and their associated 3D points to $\mathcal{F}$.
We currently refrain from automating the process of adding new keyframes because without care it can lead to drift.
Specifically, errors in the estimate pose then lead to incorrectly aligned features in $\mathcal{F}$, leading to a degradation in tracking accuracy in later frames.
\SG{The operator can add the current frame as a new keyframe by pressing a bottom on the interface: a still image of the current frame with AR is shown to the operator for validation.
If the augmentation is correctly aligned then the operator validates and otherwise discards the current frame.}

%This involves sampling many match subsets of size $4$, and for each sample creating a pose hypothesis using PnP~\cite{Lepetit2009Epnp}.
%Each hypothesis is tested for support by measuring how many of the other matches are predicted well by the hypothesis.
%Sampling and hypotheses testing is highly parallelisable, and we rely on the OpenCV's implementation.
%There are three free parameters which govern performance.
%The first is the number of hypotheses $n_k$ to sample and it represent a upper bound on the number of iterations of the RANSAC process.
%The second is the threshold $\tau_r$ (in pixels) on the reprojection error of the 3D point, below which a match is considered to support a hypothesis.
%The third is the minimum number of matches $n_c$ that must support a hypothesis.
%A larger $n_c$ means we need more matches to be confident that pose has been estimated correctly.
%We have found good default values to be $n_k=500$, $\tau_r= 12$ pixels and $n_c=15$.
%Finally, in order to compensate for the trajectory jittering we used an Extended Kalman Filter (EKF) to smooth the trajectory of the camera.




%If a pose cannot be found no pose has been found with more than $n_c$ supported matches, then we say the uterus' pose cannot be estimated for that image. 

%\subsubsection{Pose optimization and temporal smoothing}
%If enough matches support the computed pose, we then refine the pose with an efficient Gauss-Newton Optimization~\cite{Lepetit2009Epnp}.
%We use only the matches that supports the found pose and we optimize the pose using the OpenCV implementation. 
%Finally, in order to compensate for the trajectory jittering we used an Extended Kalman Filter (EKF) to smooth the trajectory of the camera.


%%%%%PROPOSE TO REMOVE THIS
%\subsubsection{Adding new keyframes}
%%In order to further stabilize and improve the pose computation, we add the possibility for an operator to add new keyframes to $\mathcal{F}$ on the fly during the live tracking. This helps to mitigate any remaining jittering and stabilizes tracking if the organ is viewed from a viewpoint that is very dissimilar to the ones present in the keyframe set.
%%%to complete the coverage of the organ from other viewpoints.
%%%When a j the user decide to add a new keyframe we use a similar procedure as described in \sect{sec:preparing}.
%%We perform this by adding the current image and its estimated pose to the keyframe set, and we add the features of the current image to $\mathcal{F}$. We currently refrain from automating the process of adding new keyframes because without care it can lead to drift. Specifically, errors in the estimate pose then leads to incorrectly aligned features in $\mathcal{F}$, leading to a degradation in tracking accuracy in later frames.

%We render the 3D model in the current camera pose to obtain a depth map of all the pixels lying within the model's silhouette.
%For each extracted feature of the current image we then compute its relevant 3D position using the depth map.
%Finally we update $\mathcal{F}$ by appending to the list the 3D feature positions, their relevant indices and descriptors.